

\chapter{Hematite Film Growth on Polycrystalline Substrates}
\label{ch:polycrystalline.growth}


\chintro{This chapter presents results for the growth of \ce{Fe2O3} films on
polycrystalline \ce{SrTiO3} substrates. Film growth on polycrystalline substrates allows
for interesting, high throughput explorations of orientation effects on film growth and
photochemical activity. The polycrystalline substrate exposes a much wider range of
orientation conditions than are available for single crystal substrates. Additionally, a
single film deposition results in thousands of individual orientation relationships. By
using electron backscatter diffraction, local orientation relationships can be observed
and determined. }


\section{Film Growth and Orientation Data Collection}
\label{sec:poly.growth.experimental}


Following the deposition parameters in \sectionref{subsec:exp.pld}, a \SI{50}{\nano\meter}
film was deposited on a polycrystalline \sto{} pellet.  The substrate pellet was prepared
as described in \sectionref{subsec:exp.solidstate}. The substrate was approximately
\SI{2}{\milli\meter} thick and \SI{8}{\milli\meter} in diameter. Depositions parameters
were as described in \tableref{pldparameters}. After deposition, an area of the film was
mapped using electron backscatter diffraction (\abbr{EBSD}). The film was polished away by
hand using \SI{0.3}{\micro\meter} colloidal silica. The film and substrate were easily
differentiated, as the film was red and the substrate was tan. Polishing was stopped when
no more red material was visible on the surface of the pellet, after
\texttildelow\SI{30}{\second}. The sample was returned to the \abbr{sem}, and the same
area of the surface was mapped. The relatively large size of the polishing abrasive used
to remove the film reduced the pattern quality of the diffraction patterns, however image
acquisition parameters could be adjusted to obtain sufficient image quality (\abbr{IQ})
and confidence index (\abbr{CI}) parameters. The data was processed with one iteration of
a grain dilation algorithm, and subsequently assigning a single average orientation to
each grain. In the grain dilation cleanup method, points not belonging to an already
identified grain (based on misorientation angle and grain size) are changed to the
orientation of the nearest grain.  For this procedure, the minimum grain size was 5 pixels
and the grain tolerance angle was 5\si{\degree}. The grain averaging procedure uses a
tolerance angle to identify grains, and then assigns a single orientation to that grain,
averaging the orientation of all points within the identified grain. The results of these
two cleanup procedures are maps with clearly identified grains of a single orientation.


\begin{figure}
\begin{center}
	\includegraphics[width=\textwidth]{subfilmmaps.pdf}
		\caption[\abbr{EBSD} maps of film and substrate]{%
			\abbr{EBSD} maps of the same area of a \SI{50}{\nano\meter}
			\ce{Fe2O3} film on a polycrystalline \ce{SrTiO3} substrate.
			Outlined areas represent the same area of the sample.}
	\label{fig:subfilmmaps}
\end{center}
\end{figure}

The substrate and film maps are shown in \figureref{subfilmmaps}. The map in
\figureref{subfilmmaps}(a) is the film, while \figureref{subfilmmaps}(b) depicts the the
same area of the substrate after film removal. The outlined regions in each map represents
pairs of film and substrate grains. For most substrate grains, a similar clearly
distinguishable set of film grains can be outlined, with borders matching the shape of the
corresponding substrate grain. In general, each substrate grain nucleates a set of film
grains, often corresponding to two distinct orientations. Even film grains that appear to
be a single red color are actually multiple film grains, as determined by the \abbr{EBSD}
software. 
\begin{figure}
\begin{center}
	\includegraphics[width=0.4\textwidth]{zoommap.pdf}
		\caption[Detail of \abbr{EBSD} map]{%
			High resolution \abbr{EBSD} detail showing the lamellar
			structure of \ce{Fe2O3} film grains within a single substrate
			grain.}
	\label{fig:zoommap}
\end{center}
\end{figure}
%\sidefigure[Detail of \abbr{EBSD} map]{%
%	High resolution \abbr{EBSD} detail showing the lamellar
%	structure of \ce{Fe2O3} film grains within a single substrate
%	grain.
%	\label{fig:zoommap}
%	}{%
%	\includegraphics[width=\marginparwidth]{zoommap.pdf}
%}{-9} 
These sets of film grains typically appear in lamellar formations, shown in the high
resolution scan detail in \figureref{zoommap}.

The colors used in inverse pole maps such as those presented in \figureref{subfilmmaps}
represent the orientation of each point on the map. The orientation assignment for each
color is determined using the color filled standard stereographic triangle. The substrate
map represents grains representing the entire color space of the key, suggesting a random
arrangement of orientations. Conversely, the inverse pole figure for the film shows grains
that are varying shades red, orange, pink, and yellow. These colors represent film grains
located near the (0001) orientation, which is represented by solid red on the inverse pole
maps. This suggests that the film grains are all nearer each other in orientation compared
to the substrate. This doesn't take into account differences between the film and
substrate phases when describing the orientation of their respective grains. Because of
its higher symmetry, the entire range of orientations of the cubic substrate can be
represented in a smaller standard stereographic triangle than the hexagonal film material.
The angle between the red, (001)-oriented and blue, (111)-oriented substrate grains (the
maximum misorientation for cubic crystals) is \SI{54.7}{\degree}. For the hexagonal \feo,
the angle between red grains and blue grains is \SI{90}{\degree}. 
\begin{figure}
	\begin{center}
	\includegraphics[width=0.5\textwidth]{stereotriangles.pdf}
		\caption[Relationship of stereographic triangles]{%
			Relationship of the standard stereographic triangles for hexagonal
			and cubic systems to each other and the complete stereographic 
			projection.}
	\label{fig:stereotriangles}
	\end{center}
\end{figure}
%\sidefigure[Relationship of stereographic triangles]{%
%	Relationship of the standard stereographic triangles for hexagonal
%	and cubic systems to each other and the complete stereographic 
%	projection.
%	\label{fig:stereotriangles}
%	}{%
%	\includegraphics[width=\marginparwidth]{stereotriangles.pdf}
%}{-15}
\figureref{stereotriangles} shows the relationship of the cubic and hexagonal
stereographic triangles to each other and the entire stereographic projection. The same
color scale represents a wider set of orientations in the hexagonal system. As a result, a
simple comparison of the color variance between these two images doesn't accurately
reflect the differences in orientation spread for the two maps.


\section{Orientation Relationship Analysis}
\label{sec:poly.growth.orients}


For \ce{Fe2O3} films on single crystal \ce{SrTiO3} (001) substrates, the orientation
relationship was \ce{Fe2O3}(0001)||\-\ce{SrTiO3}(111) for a significant portion of film
grains (\texttildelow90\%), even though the \ce{SrTiO3} (111) direction was not the out of
plane direction. It was hypothesized that a similar relationship could be determined for
the substrate grains tilted away from the (111) direction on the polycrystalline
substrate. The \ce{Fe2O3}(0001)||\ce{SrTiO3}(111) orientation relationship would persist,
even for grains tilted away from (111) orientation. To test this hypothesis, data from
these maps was exported in the form of a text file containing a single line for each
identified grain of the scan. Each grain was assigned a unique number ID. This ID was
included, along with the Euler angles corresponding to the angle of that grain.  The grain
ID for each film grain was manually paired with the grain ID for its corresponding
substrate grain. This was done through visible inspection of the maps depicted in Figure
\ref{fig:subfilmmaps}. Only film grains that could clearly be assigned to a substrate
grain were included in the pairing list. Because multiple film grains exist on a single
substrate grain, in many cases multiple pairings exist for a single substrate grain. A
program was then used to calculate the minimum angle between a film direction and a
substrate direction, taking into account symmetry operations for the film and substrate
crystal structures. 

117 such film/substrate pairs were analyzed to determine their orientation relationships.
Because each substrate grain corresponds to multiple film grains, the 117 substrate grains
represent 501 distinct film grains. \figureref{subfilmipfs} shows standard stereographic
triangles for the film and substrate, with each point within the triangles representing a
substrate or film grain used for orientation relationship calculations.
\begin{figure}
	\includegraphics[width=\textwidth]{subfilmipfs.pdf}
		\caption[Orientation of film and substrate grains]{%
			Standard stereographic triangles representing the orientations
			of all grains in \figureref{subfilmmaps}. Each point on the 
			triangles represents a grain used for orientation calculations.
			Shaded points in the hexagonal film triangle represent film grains
			with an outlier ``out of plane'' orientation relationship.}
	\label{fig:subfilmipfs}
\end{figure}
The substrate grains range widely over the entire area of the cubic standard stereographic
triangle. The film grains are clustered near the (0001) point of the triangle. Once again,
the same caveat as for the inverse pole maps comes into play. The angle between the (001)
point and the edge for the cubic triangle ranges from \SI{45}{\degree} for the (101)
corner and \SI{54.7}{\degree} for the (111) corner. Then angle between the (001) point and
all edge points for the hexagonal triangle is \SI{90}{\degree}. The angular spread of the
points represented in the film triangle is close to the same angular spread represented by
the entire area of the cubic substrate triangle.
\begin{table}
	
	\begin{center}
\begin{tabular}{lr}
	
		Min &
		0.26 \\
		
		Q1 &
		1.86 \\
		
		Median &
		2.77 \\
		
		Q3 &
		3.64 \\
		
		Mean &
		4.22 \\
		
		Std. Dev. &
		6.50 \\
		
		Lower Fence &
		-3.46 \\
		
		Upper Fence &
		8.96 \\
		
	\end{tabular}

\end{center}	\caption[Statistical descriptors of orientation relationship
data]{Statistical descriptors of \feo[0001]/\sto[111] orientation relationship data.}
	\label{tab:outofplanestats}


\end{table}

%\sidetable[Statistical descriptors of orientation relationship data]{%
%	Statistical descriptors of \feo[0001]/\sto[111] orientation relationship data.
%	\label{tab:outofplanestats}}{%
%	\vspace{-2in} %to move the table
%	\begin{tabular}{lr}
%	
%		Min &
%		0.26 \\
%		
%		Q1 &
%		1.86 \\
%		
%		Median &
%		2.77 \\
%		
%		Q3 &
%		3.64 \\
%		
%		Mean &
%		4.22 \\
%		
%		Std. Dev. &
%		6.50 \\
%		
%		Lower Fence &
%		-3.46 \\
%		
%		Upper Fence &
%		8.96 \\
%		
%	\end{tabular}
%}
When the angle between the substrate [111] direction and the film [0001] direction is
calculated for the 501 identified pairs in these scans, the average angle between the
substrate [111] and the film [0001] is \SI{4.22}{\degree}. If outliers, defined as
\begin{gather}
\text{Lower Outlier} < \text{Lower Fence} = Q_{1}-3(Q_{3}-Q_{1})\\
\text{Upper Outlier} > \text{Upper Fence}=  Q_{3}+3(Q_{3}-Q_{1})
\end{gather}
where $Q_{1}$ is the lower quartile, $Q_{3}$ is the upper quartile, and $Q_{3}-Q_{1}$ is
the interquartile range, the average angle is \SI{2.62}{\degree}.
\tableref{outofplanestats} lists the statistical descriptors of the orientation
relationship calculations. For the majority of substrate/film pairs, the film c-axis grew
parallel to the substrate [111] direction, regardless of that direction's angle away from
the surface normal. This orientation relationship is labeled as the ``out of plane''
relationship, mirroring the results for film growth on single crystal (111) oriented \sto
substrates.

The same orientation program was used to calculated the angle between the substrate [110]
direction and the film [10\={1}0] direction. This proposed relation reflects observed
alignment of the prismatic directions for films grown on single crystal substrates. The
average angle between the calculated directions was \SI{2.45}{\degree}. By the same
analogy to growth on \sto (111) substrates as for the \ce{Fe2O3}[0001]||\ce{SrTiO3}[111],
this alignment of the film [10\={1}0] direction and the substrate [110] direction is
labeled as the ``in plane'' relationship.

\begin{figure}

	\begin{center}
\includegraphics[width=0.6\textwidth]{orplots.pdf}
		\caption[Plot of film-substrate misorientations]{%
			Plot of the angle of misorientation between the film
			[0001] direction and the substrate [111] direction (out of plane, 
			blue) and the film [10\={1}0] direction and substrate [1\={1}0] 
			direction (red).}
	\label{fig:orplots}
\end{center}
\end{figure}
\figureref{orplots} tabulates the results for the orientation relationship calculations
including all calculations for both alignments, the ``out of plane''
\ce{Fe2O3}[0001]||\ce{SrTiO3}[111] and the ``in plane''
\ce{Fe2O3}[10\={1}0]||\ce{SrTiO3}[1\={1}0]. For each plot, the majority of calculated
relationships fall within the calculated upper fence. In general, points that were
outliers (beyond the upper fence) for the ``out of plane'' relationship were also outliers
for the ``in plane'' relationship. It is important to note here that the description of
these points as outliers does not imply an error in data collection, or that the values
calculated for these points don't represent the actual orientation relationships. This is
shown here for six example substrate grains, on which outlier film orientation were found.
Each identified film grain had consistent ``out of plane'' and ``in plane'' orientation
relationships. These relationships are listed in \tableref{weirdgrains}.
\begin{table}
	
\centering
	\begin{tabular}{rrrr}

Substrate ID & Film ID &  [0001]$_{\text{Film}}$/[111]$_{\text{Sub}}$
(\degree)		&	[10\={1}0]$_{\text{Film}}$/[110]$_{\text{Sub}}$ (\degree)		\\

\cmidrule(lr){1-1}
\cmidrule(lr){2-2}
\cmidrule(lr){3-3}
\cmidrule(lr){4-4}

652		&	645		&	37.59	&	18.39	\\
		&	747		&	37.62	&	18.35	\\
		&	764		&	37.88	&	18.16	\\[4pt]
804		&	914		&	32.45	&	25.11	\\
		&	915		&	32.52	&	25.20	\\
		&	1020	&	32.67	&	25.21	\\
		&	922		&	32.73	&	25.22	\\[4pt]
828		&	1081	&	12.10	&	25.74	\\
		&	1089	&	12.80	&	26.11	\\
		&	1100	&	12.84	&	26.13	\\[4pt]
946		&	1409	&	33.78	&	8.59	\\
		&	1422	&	38.32	&	16.55	\\
		&	1393	&	38.43	&	12.53	\\
		&	1395	&	39.41	&	12.75	\\[4pt]
1468	&	2044	&	19.73	&	20.29	\\
		&	2058	&	19.79	&	20.48	\\
		&	2175	&	19.84	&	20.60	\\
	
	\end{tabular}

	\caption[Outlier misorientations]{%
		List of misorientation angles for grains that do not follow
		the observed orientation relationship between substrate and 
		film.}
	\label{tab:weirdgrains}
\end{table}
While within each grain, the orientation relationship is consistent, there is no single
alignment that would describe all of these points. Three substrate grains %652, 804, and 946
promoted films with the c-axis angled 30-40\degree from the substrate [111] direction.
Two additional grains prompted film grains with a \texttildelow\ang{12} misorientation and
a \texttildelow\ang{20} misorientation. Similarly, the ``in plane'' orientation relations
are generally consistent within each substrate grain, but no single orientation
relationship describes all the outlier grains.





\section{Discussion}
\label{sec:poly.growth.discussion}
		
		
Similar to the previously discussed growth of polycrystalline \ce{Fe2O3} films on
\ce{SrTiO3} (001) substrates, the orientation relationship between the substrate and film
is more complicated than the typical lattice parameter or interface plane view of epitaxy.
A 2\textsc{d} description of the lattice matching wouldn't necessarily predict the
consistent orientation relationship seen over a wide range of substrate orientations.
Similarly, a description of the orientation relationship relying solely on out of plane
and in plane vectors in the sample reference frame doesn't show the underlying consistent
orientation relationship for these films on polycrystalline substrates. If the
conventional sample normal vector were to be used to describe each substrate grain
orientation relationship, the result would be 117 different orientation relationship
descriptions, rather than the one consistent descriptor from the lattice reference frame
used here to describe the majority of film-substrate pairs.

Across wide ranges of substrate orientations, the orientation relationship of the film
[0001] and [10\={1}0] directions aligned with the substrate [111] and [1\={1}0] directions
respectively. This relationship matches that observed for films on (111)-oriented single
crystal substrates, as well as for 2 of the observed classes of film grains on
(001)-oriented single crystal substrates (the purple and white partitions reported in
\sectionpageref{subsubsec:single.growth.purplewhite}.) Based on single crystal substrate
results, grains oriented very near the (111) orientation would be expected to show this
relationship from a simple 2\textsc{d} lattice parameter analysis.\footnote{See
\sectionpageref{sec:single.growth.111}.}

An analysis of the close packed (eutactic)\cite{OKeeffe:1977vx} crystal directions of the
film and substrate leads to a proposed mechanism for the persistent orientation
relationship. The hematite structure is a hexagonal close packed (hcp) network of oxygen
atoms, with iron atoms filling \slantfrac{2}{3} of the interstitial sites. The close
packed oxygen plane is the (0001) plane, and the close packed direction within this plane
is the <1\={1}00> family of directions. The \ce{SrTiO3} substrate is a cubic close packed
(ccp) network of \ce{SrO3} atoms, with titanium atoms in \slantfrac{1}{4} of the
octahedral interstitial sites. The close packed planes are the \{111\} family of planes,
and the close packed directions within that plane are <1\={1}0>. The close placed planes,
\ce{Fe2O3}[0001] and \ce{SrTiO3}[111], are also the planes that represent the ``out of
plane'' orientation relationship. The close packed planes are consistently aligned between
substrate and film grains. The close packed directions match the determined ``in plane''
orientation relationship. For \ce{Fe2O3} films grown on \ce{SrTiO3} substrates, the close
packed network is aligned between film and substrate over a wide spread of substrate
orientation conditions.

It is noted that the use of multiple film/substrate pairs for each substrate grain affects
the validity of the statistical analysis of the data. Because each substrate grain
resulted in an average of \texttildelow{}4 calculated orientation relationships, much of
the data can be thought to represent redundant information. Each additional film grain
doesn't represent an entirely unique substrate-grain pair. Despite this, the decision was
made to include all data for orientation relationship calculations. In all cases, at least
two different film orientation bins were observed on a single substrate grain, regardless
of the actual number of film grains. Manually selecting just one of these orientation
families could leave out important differences between the film grains. Additionally, in
some cases, some film grains showed a very small misorientation between the selected
substrate and film directions, while other film grains \emph{on the same substrate grain}
showed a much larger misoriention. If only one film grain was selected for each substrate
grain, this effect could be amplified or completely missed, depending on the random
selection of film grain. Finally, the inclusion of all film/substrate orientation
calculations provides a higher sample size. The larger sample size will result in a more
accurate statistical depiction of the total population, so long as the limitations here
are understood.

The use of polycrystalline substrates and electron backscatter diffraction mapping for
film growth analysis represents a technical step forward. The combination of these
techniques, called combinatorial substrate epitaxy (\abbr{CSE}), allows for high
throughput studies of film growth on high and low index orientations. With the inclusion
of a local probe property measurement technique, such as the marker reactions used in this
document or \abbr{AFM} probing of electronic properties, effects of orientation on
material properties can be investigated. Using \abbr{CSE}, a single film deposition
results in hundreds of individual substrate/film pairs. Each of these pairs can be thought
to represent a single growth experiment. Even if single crystal substrates could be
obtained in all of these orientations, the deposition time alone would prohibit such
wide-ranging epitaxy analyses.


\section{Conclusion}
\label{sec:poly.growth.conclusion}

\ce{Fe2O3} films were grown on polycrystalline \ce{SrTiO3} substrates. The film-substrate
orientation relationship was consistent for most substrate grains. The close packed
(eutactic) network of the film grains aligned with that of the substrate grains. For the
\ce{Fe2O3}/\ce{SrTiO3} system, this results in an arrangement described by
\ce{Fe2O3}[0001]||\ce{SrTiO3}[111] and \ce{Fe2O3}[10\={1}0]||\ce{SrTiO3}[1\={1}0]. This
orientation relationship is the same as observed for films on (111)-oriented single
crystal substrates, however in this case, the (111) plane is not parallel to the surface
for most substrate crystallites. The eutactic orientation relationship persisted, even
when the close packed plane was tilted away from the surface normal.

