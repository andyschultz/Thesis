

\chapter{Conclusions \oldand Future Work}
\label{ch:conclusions}


\section{Summary of Photochemical Activity
Results}\label{sec:conclusions.photochemc.summary}


The visible light photochemical activity of \ce{Fe2O3} in different structures has been
studied. Anisotropic photochemical reactivity was observed for bulk hematite polycrystals. 
Two identical \ce{Fe2O3} films supported on different substrates showed highly
different photochemical activities. The film supported on the \ce{SrTiO3}(111) surface was
much more reactive than the film on Al2O3(0001). These films were even more reactive than
for bulk \ce{Fe2O3} polycrystals, even though significantly less light was absorbed in the
film structure, owing to the thinness of the film. 

The same was true for films on polycrystalline \ce{SrTiO3} substrates. The reactive grains
of films on polycrystalline substrate were significantly more reactive than the bulk
material. The orientation dependence of reactivity for these films echoed that of bulk
\ce{Fe2O3}, but increased the range of orientations that promoted reactive grains. For the
bulk material, only grains located relatively near the (1\={2}10) orientation were
observed to be reactive. For the film, grains significantly farther from this orientation
were observed to be reactive, and grains that were completely nonreactive in the bulk
material were at least moderately reactive for the film.

The orientation dependence of the photochemical reactivity of \ce{Fe2O3} itself is a new
observation. Previous studies of anisotropic photochemical activity for hematite crystals
were much narrower in scope, focusing only on differentiating basal and prismatic
surfaces. The work in this document was able to differentiate levels of reactivity between
different prismatic surfaces, as well as the reactivity of high index surfaces.
Additionally, Kelvin probe force microscopy showed that the surface potential of reactive
grains was higher than that for nonreactive grains.


\section{Summary of Film Growth Results}
\label{sec:conclusions.growth.summary}


In the process of examining the photochemical activity of \ce{Fe2O3} films, \ce{Fe2O3}
film growth was also studied. Electron backscatter diffraction was used to map the
orientation of substrates and their corresponding films. From this data, orientation
relationships between the substrate and film were determined. Polycrystalline films on
\ce{SrTiO3}(001) substrates showed unusual texture. It was determined that for a
significant portion of film grains, the grain orientation was a result of the alignment of
crystal directions far away from the surface normal. For \ce{SrTiO3}(001) substrates, the
alignment of the substrate [111] direction with the film [0001] direction was observed.
The corresponds with the orientation relationship observed on \ce{SrTiO3}(111) substrates.
The idea of epitaxy on high index orientations was also examined through \abbr{EBSD}
studies of film growth on polycrystalline substrates. For films on polycrystalline
substrates, it was observed that the film accepted its orientation according to the
eutactic arrangement of the substrate grains. As the arrangement of the close packed
network of the substrate tilted in space, the arrangement of the close packed network of
the film followed. The close packed network of the film consistently lined up with that of
the substrate over a wide spread of substrate orientations.

Together, this work represents a step forward in the understanding of mechanisms for
creating charged interfaces. It also represents progress in the incorporation of visible
light active materials into these heterostructures. In the experiments presented in this
document, visible light illumination drives most photochemical reactions. Charge carriers
are only generated in the film, different from earlier studies on heterostructures with
charged interfaces. \abbr{EBSD} has previously been used to study the orientation
relationships between substrate and film. However the analysis presented in this document
is significantly more complicated. Unlike earlier studies, entire maps of film
orientations could be obtained. This allows for the analysis of a much larger dataset. The
study of polycrystalline film growth and texture on single crystal substrates, and the use
of \abbr{EBSD} to measure epitaxy at high index orientations is an interesting new
development.

\section{Summary of Bismuth Ferrite Results}
\label{sec:conclusions.bfo.summary}

\ce{BiFeO3} surfaces exhibit spatially selective visible-light photochemical activity.
Silver ions in solution were photochemically reduced by the \ce{BiFeO3}, depositing solid
silver on the surface in patterns corresponding to positive ferroelectric domains. Upward
band bending in the negative domains prevents electrons from reaching the surface and
these locations do not reduce silver. Electric fields arising from ferroelectric domains
at the surface overwhelm anisotropy in the photochemical activity that might arise from
grain orientation alone.

\section{Future Paths}
\label{sec:conclusions.future.paths}

The work presented in this document provides numerous research paths for future
exploration. The results presented here are not entirely consistent with the hypothesis 
that the presence of
polar surface terminations are responsible for the increased photochemical activity. The
comparison of photochemical activity for \ce{Fe2O3} films on \ce{SrTiO3}(001) and
\ce{SrTiO3}(111) surfaces is not yet complete. This comparison removes other possibilities
for the explanation of the higher reactivity of films on \ce{SrTiO3}(111) substrates. The
difficulty in this comparison lies in the differing microstructures of films on
\ce{SrTiO3}(001) and \ce{SrTiO3}(111). This comparison would test the hypothesis that the
polar \ce{SrTiO3}(111) surface is responsible for the increased reactivity. Aditionally,
experiments designed to explicitly isolate the presence and effects of polar surfaces
would help reinforce this interpretation of the results reported within this document.

The use of electron backscatter diffraction  and polycrystalline substrates to perform
combination substrate epitaxy (CSE) experiments is still in its infancy. So far, this
technique has been used to study \ce{TiO2} film growth on \ce{BaTiO3} and \ce{BiFeO3} and,
within this document, \ce{Fe2O3} films on \ce{SrTiO3} substrates. These leave a huge
range of film and substrate materials open for future exploration with this technique.
Additionally, the idea of eutactic epitaxy arrangements offers an opportunity to predict
and develop models for film growth on high index surfaces or for complex heteroepitaxial
systems. Future results could be used to develop a model for the thermodynamic and kinetic
factors influencing the relative rates of nucleation and growth for eutactic orientations.

Finally, the use of the marker reactions in this document provides initial evidence
regarding the photochemical activity of the materials presented in this document. However,
the marker reactions are significantly different in nature than those involved in the
practical applications of photochemically active materials. Implementing conventional
photoelectrochemical testing methods for the analysis of the heterostructures presented in
this document would help compare the structures here to other reports of photochemical
activity. 
