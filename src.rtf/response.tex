\documentclass[12pt,%
               %draft,%
               letterpaper,
               oneside]{uiothesis}
\usepackage{xr}
\externaldocument{base}                  
\usepackage{enumitem}    
           
\begin{document}
  \chapterstyle{uio}
  \pagestyle{uio}

\chapter*{Overview Feedback Response}

%\OnehalfSpacing

\begin{enum}
  \item \emph{The most important focus for your thesis is to execute your plan to
understand the photochemistry of the heterostructures; you must resolve the effects of
orientation interface charge, and microstructure that hinder our current
understanding.}\vspace{8pt}
  
	New research on the orientation dependence of photochemical reactivity for
polycrystalline hematite samples has been carried out, and is reported in
\chapterpageref{fe2o3orientation}. These results, along with new results for films on
polycrystalline \ce{SrTiO3} supports in
\chapterpageref{polycrystalline.reactivity}.\vspace{16pt}
  
  \item \emph{An essential component of resolving the photochemical properties of the
heterostructures is to carry out the work on polycrystalline \ce{SrTiO3} supports. The use
of \ce{BaTiO3} supports only clouds the issue currently.}\vspace{8pt}
  
  \chapterpageref{polycrystalline.reactivity} has been completely rewritten to reflect
work carried out on polycrystalline \ce{SrTiO3} supports rather than \ce{BaTiO3} supports.
The new results generally support previous assertions for \ce{BaTiO3} supports, but allow
for increased clarity when relating the results to those on single crystal substrates.
Additionally, higher quality film growth was observed on the \ce{SrTiO3} support, allowing
for better analysis of orientation effects on reactivity.\vspace{16pt}
  
  \item \emph{While your work aimed to characterize the epitaxial relationships of
\ce{Fe2O3} and \ce{SrTiO3} is interesting, this must not serve as the central focus for
your remaining work. If time permits during or after the resolution of the photochemical
properties, this work will make an interesting but non-essential addition to your
thesis.}\vspace{8pt}
  
  As suggested, no additional film growth experiments regarding the epitaxial
relationships were performed. However, the data already obtained and presented for the
thesis overview were further analyzed. The role and relationship of close-packed oxygen
planes and directions in determining the orientation relationships between film and
substrate were further clarified, and \sectionpageref{subsec:single.growth.discussion} in
\chapterref{single.crystal.growth} has been updated to reflect this new
information.\vspace{16pt}
  
  \item \emph{Be careful to consider the effect of grain size (microstructure) on
reactivity, especially when using intermediate sized grains in a polycrystalline film
microstructure, ensuring they can be properly compared to other observations.}\vspace{8pt}
  
  Grain size effects were taken into consideration in interpreting the results for
photochemistry on polycrystalline films, particularly in the case of films on
polycrystalline substrates. \vspace{16pt}
  
  \item \emph{You should discuss how the surface roughness of the differently oriented
grains might affect your results on reactivity, your ability to index patterns, or film
growth.} \vspace{8pt}
  
  \sectionpageref{subsec:background.surfaceactivity} has been added to the thesis
document. This section addresses roughness affects in the context of photochemical
activity.\vspace{16pt}
  
  \item \emph{You should put units on  your pole figures and discuss whether the
distributions accurately describe reality. It might be of interest to include a reasonable
RD [rolling direction] IPF to show the alignments in the specific subgroups.} \vspace{8pt}
  
  All pole figures presented in the thesis overview document included units. When
introducing the concept of pole figures in \chapterref{single.crystal.growth}, comments on
the units of the figures have been added. Additional comments on how the specific scale of
the pole figures relates to reality have been added where appropriate. In the oral
presentation, units have been added for all presentations of pole figures and inverse pole
figures.\vspace{16pt}
  
  \item \emph{In your document, you need to be careful to treat the work such that all the
related communities will be satisfied it was carried out in a reasonable fashion:
photochemistry, catalysis, electrochemistry, materials science, film growth, electronic
materials. In this bullet, a short list of items that were brought up are given to be
addressed.} \vspace{8pt}
  
  \begin{enumerate}[label=\alph*,leftmargin=1em]
  \item \emph{Background on the activity of a solid-state surface in photochemistry should
be given, with a focus on both necessary reactions. This should serve as a guide to what
structural aspects you can control and what approach you are using.}\vspace{16pt}
  
    \chapterref{background} has been updated to include information on factors at the
surface that affect photochemistry of semiconductors.\vspace{16pt}
  
  \item \emph{A similar background should be given on optimizing the length scales or
spatial metrics.}\vspace{8pt}
  
  \chapterref{background} has been updated to include
\sectionpageref{subsec:background.lengthscales}, which discusses relevant length scales in
semiconductor photochemistry. Where appropriate, discussions of length scale
considerations are included within each experimental results chapter.\vspace{16pt}
  
  \item \emph{The thesis should include discussions on how you treat, and why, the roles
of pH and the counter-reaction on your observations.}\vspace{8pt}
  
  \sectionref{subsec:exp.markerreactions} has been updated to include a discussion of the
effects of pH on band edge positions, and the relevance to the work presented in this
document.
  
  \vspace{16pt}
  
  \item \emph{Include a discussion of mass transfer limitations, or your ability to ignore
them.}\vspace{8pt}
  
  The experimental description of the marker reactions has been updated to include a
comment on mass transfer on p.~\pageref{masstransfer}, with an explanation as to why any
effects of mass transfer on the rate of reaction were discounted for the experiments
presented in this document.\vspace{16pt}
  
  \item \emph{Include a discussion, at least, of the effects of local pH (or Pourbaix)
shifts and their effects on your observations.}\vspace{8pt}
  
  \sectionref{subsec:exp.markerreactions} has been updated to include a discussion of the
effects of pH on band edge positions, and the relevance to the work presented in this
document.
 
  \vspace{16pt}
  
  \item \emph{Include a discussion as to how initial reactivity and long-time reactivity
might be different owing to the existence of previously deposited solid material. Again,
focus on how your results are impacted by this.}\vspace{8pt}
  
  The section on photochemical marker reactions has been expanded on p.
\pageref{timescales} to include a discussion of time scales and possible associated
complications to interpretation of results.
  
  \vspace{16pt}
  
  \item \emph{Be consistent with usage of scientific terms. For example, if photocatalysis
implies a barrier or overpotential reduction, then how are you addressing
photocatalysis?}\vspace{8pt}
  
  The document has been edited to verify consistent usage of scientific terms.
  
  \vspace{16pt}
  
  \item \emph{You should clarify the STM experiment discussion that indicates the material
is p-type with standard contact energy level diagrams of the experiment.}\vspace{8pt}
  
  Energy level diagrams illustrating the theory behind the STM experiment have been added
to \sectionref{sec:ch7results}.\vspace{16pt}
  
  \item \emph{You should include a proper discussion of how minority phases were assigned
to features in an AFM image; clarify how you made the assignment to specific
features.}\vspace{8pt}
  
  A brief discussion of how minority phase regions were assigned has been added to
\sectionref{sec:ch7results} on p. \pageref{minorityphase}.\vspace{16pt} 
  
  \item \emph{The description of the epitaxy findings, their significance, and relation of
existing work should be improved.}\vspace{8pt}
  
  Throughout the document, discussions of epitaxy findings have been expanded.
  
  \vspace{16pt}
  
\end{enumerate}
  
\end{enum}



\end{document}