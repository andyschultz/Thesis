\chapter{Abstract}

For photochemical hydrogen production to reach acceptable efficiencies, semiconductor photolysis systems that make use of visible light must be developed. This work presents results for the photochemical activity of iron-based materials and structures. Hematite, or \textalpha-\ce{Fe2O3}, and \ce{BiFeO3} absorb light in the visible range. 

The photochemical reactivity of bulk \ce{Fe2O3} and thin \ce{Fe2O3} films on single crystal and polycrystalline substrates is reported. Bulk \ce{Fe2O3} crystallites show strong anisotropic photochemical activity. Crystallites with a surface orientation near the hexagonal (1\={2}10) plane are significantly more reactive than other orientations. Thin \ce{Fe2O3} films supported on \ce{SrTiO3} (111) substrates are significantly more reactive than similar films on \ce{Al2O3} substrates and bulk polycrystalline hematite. Films on polycrystalline substrates showed similar orientation dependent reactivity as for bulk \ce{Fe2O3} crystallites, with overall higher reactivity than the bulk material. Overall, films supported on \ce{SrTiO3} substrates were more reactive than the bulk material and films on \ce{Al2O3} substrates.

Details on the growth via pulsed laser deposition of \ce{Fe2O3} films on perovskite \ce{SrTiO3} substrates are also reported. Orientation relationships between films and single crystal or polycrystalline substrates were determined using electron backscatter diffraction. Epitaxial (0001)-oriented films were grown on \ce{SrTiO3} (111) substrates. Films on \ce{SrTiO3} (001) substrates were polycrystalline, but showed preferred orientations based on alignment of close-packed (eutactic) networks. Films on polycrystalline \ce{SrTiO3} substrates also showed alignment of eutactic networks between substrate and film. The heteroepitaxial growth of polycrystalline films on polycrystalline substrates presents a much wider range of orientation conditions than available for growth on single crystals. Combinatorial substrate epitaxy, the growth and analysis using electron backscatter diffraction of films on polycrystalline substrates, opens many opportunities for wide-ranging epitaxy analyses.

The photochemical behavior of \ce{BiFiO3} surfaces is reported. BiFeO3 surfaces exhibit spatially selective visible-light photochemical activity. Silver ions in solution were photochemically reduced by the BiFeO3, depositing solid silver on the surface in patterns corresponding to positive ferroelectric domains. This is suggested to arise from upward band bending in the negative domains prevents electrons from reaching the surface and these locations do not reduce silver. Electric fields arising from ferroelectric domains at the surface overwhelm anisotropy in the photochemical activity that might arise from grain orientation alone.