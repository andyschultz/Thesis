% !TEX root = base.tex 

\chapter{Hematite Film Reactivity on Polycrystalline Substrates}
\label{ch:polycrystalline.reactivity}


\chintro{This chapter reports on the photochemical activity of the hematite films supported on polycrystalline \ce{SrTiO3} substrates. The area examined for reactivity experiments was the same area used to determine the orientation relationships, discussed later in \chapterref{polycrystalline.growth}. After reaction with the silver nitrate solution, the reactivity of film grains was correlated to substrate and film reactivity.}


\section{Experimental Details}
\label{sec:poly.reac.experimental}


The photochemical activity of \ce{Fe2O3} films grown on polycrystalline \ce{SrTiO3} substrates is reported in this chapter. An analysis of film growth and orientation relationships is reported later in \chapterref{polycrystalline.growth}. Details on film growth can be found in \sectionpageref{sec:poly.growth.experimental}. Electron backscatter diffraction (\abbr{EBSD}) data from \figureref{subfilmmaps} was used to correlate reactivity with orientation. Details for data collection can be found in \sectionpageref{sec:poly.growth.experimental}. Before the removal of the film to analyze substrate grain orientations, the photochemical marker reaction was performed on the film and analyzed using optical and atomic force microscopy (\abbr{AFM}). Marker reactions were performed as described in \sectionpageref{subsec:exp.markerreactions}. The light source was a blue \abbr{LED}, and the reaction time was one minute. This time was selected after trials with reaction times, and was found to result in some grains that were bare, while other grains were highly reactive. Longer reaction times resulted in the entire surface covered in reaction product. After these longer reaction times, silver colored material visibly coated the reaction surface, without the aid of microscope analysis. Because the light source was the \abbr{LED}, only visible light was available to excite electrons to the conduction band. Because the band gap of \ce{SrTiO3} requires ultraviolet light to generate charge carriers, all photochemical activity was attributed to the film. It was assumed that a negligible number of charge carriers were generated in the substrate.

\section{Results}
\label{sec:poly.reac.results}

After reaction with the silver nitrate solution, the film was examined using optical and atomic force microscopy. Optical microscopy provided a rapid, high throughput classification of the entire area. \abbr{AFM} then confirmed the interpretation of the optical micrograph, and provided a detailed look at small subsections of the examined area.


\subsection{Optical Microscopy}
\label{subsec:poly.reac.optical}


\figureref{filmoptical} shows an optical micrograph of an area of the sample surface after reaction. Dark areas on the micrograph generally indicate areas with a large amount of reaction product. Pores are also responsible for some areas of dark contrast. Any pores could be identified by comparison with an image of the clean sample surface. The image of the clean surface also served as a reference, verifying that dark contrast was not present. The clean surface is uniformly bright, with the exception of pores, cracks, and grain boundaries. These areas appear as dark spots on the micrograph. Some grains appear slightly darker than others. This is a result of differing interaction with the polarized light, and does not represent reaction product on the surface. \figureref{filmclean} shows the surface after reaction product was cleaned, for comparison with the image in \figureref{filmoptical}.
\begin{figure}
	\includegraphics[width=0.8\textwidth]{filmoptical.pdf}
		\caption[Film surface after reaction]{%
			Surface of \ce{Fe2O3} film on polycrystalline
			\ce{SrTiO3} substrate after marker reaction. Areas
			of dark contrast correspond to portions of the surface
			covered with a large amount of reaction product.
			Bright areas are bare areas, with little reaction
			product.}
	\label{fig:filmoptical}
\end{figure}
%
%
\begin{figure}
\begin{center}
\includegraphics[width=0.8\textwidth]{filmclean.jpg}
\caption[Cleaned film surface]{%
	Surface of \ce{Fe2O3} film on polycrystalline \ce{SrTiO3} 
	substrate after cleaning. The surface shows uniform contrast, exception pores, cracks, and the affect of light polarization,
	supporting the interpretation of contrast in \figureref{filmoptical}.}
\label{fig:filmclean}
\end{center}
\end{figure}
%\sidefigure[Cleaned film surface.]{%
%	Surface of \ce{Fe2O3} film on polycrystalline \ce{SrTiO3} 
%	substrate after cleaning. The surface shows uniform contrast,
%	supporting the interpretation of contrast in \figureref{filmoptical}.
%	\label{fig:filmclean}
%	}{%
%	\includegraphics[width=\marginparwidth]{filmclean.jpg}
%}{0} % Last argument is number of lines to move up or down

Using the electron backscatter diffraction maps in \figureref{subfilmmaps} as a guide, each individual film grain was identified in the optical image in \figureref{filmoptical}. The identified grains were manually classified as highly reactive, moderately reactive, or nonreactive. Grains that were uniformly dark were classified as highly reactive. Conversely, grains that were uniformly bright were nonreactive. Grains that appeared a medium gray color, or were otherwise not uniformly covered by reaction product were labelled moderately reactive.

\figureref{labeledgrains} shows a digram of the resulting classifications. The grain boundaries were taken from the \abbr{EBSD} map of the substrate. Dark blue portions of this map correspond to highly reactive grains, medium blue to moderately reactive grains, and light blue areas are nonreactive grains. White corresponds to areas where no assignment was made. This occurred when it was unclear the reactivity level of a grain, owing to image artifacts or difficulty distinguishing grains. Also, some white grains in \figureref{labeledgrains} were outside the field of view of the microscope image in \figureref{filmclean}.
\begin{figure}
	\includegraphics[width=\textwidth]{labeledgrains.pdf}
		\caption[Reactivity assignments for film]{%
			Grain boundary network taken from \abbr{EBSD} data in 
			\figureref{subfilmmaps}(b) with assignments of reactivity
			observed in \figureref{filmclean}. Dark blue represents 
			highly reactive grains, medium blue represents moderately
			reactive grains, and light blue represents nonreactive 
			grains. White grains are those that were not assigned a 
			reactivity level.}
	\label{fig:labeledgrains}
\end{figure}

The orientation of the substrate and film grains was plotted on the standard stereographic triangles for cubic and hexagonal crystal structures respectively. These plots are depicted in \figureref{substrateplot} and \ref{fig:filmplots}. Both film and substrate orientations were examined to determine whether any reactivity differences were a result of substrate or film orientation alone, or a combination of the two.

\begin{figure}
	\begin{center}
	\includegraphics[width=0.6\textwidth]{substrateplot.pdf}
	\caption[Orientation of substrate grains]{%
	Standard cubic stereographic triangle showing the orientation of
	substrate grains classified in \figureref{labeledgrains}. Dark points are
	highly reactive grains, light points are moderately reactive grains,
	and empty points are nonreactive grains.}
	\label{fig:substrateplot}
	\end{center}
\end{figure}
%\sidefigure[Orientation of substrate grains]{%
%	Standard stereographic triangle showing the orientation of
%	grains classified in \figureref{labeledgrains}. Dark points are
%	highly reactive grains, light points are moderately reactive grains,
%	and empty points are nonreactive grains.
%	\label{fig:substrateplot}
%	}{%
%	\includegraphics[width=\marginparwidth]{substrateplot.pdf}
%}{-8} % Last argument is number of lines to move up or down
In the plot of the substrate grains in \figureref{substrateplot}, highly reactive grains are represented by dark points, moderately reactive grains by light points, and nonreactive grains by empty points. The figure doesn't show a strong relation between substrate orientation and grain reactivity. Highly reactive and moderately reactive points are spread throughout the standard stereographic triangle. Nonreactive grains are weakly clustered near the (111) orientation and along the axis between the (001) and (101) orientations.

The reactivity of the film grains are plotted on the three hexagonal triangles in \figureref{filmplots}. 
\begin{figure}2
	\includegraphics[width=\textwidth]{filmplots.pdf}
		\caption[Orientation of film grains]{%
			Standard stereographic triangles showing the orientation
			of (a) highly reactive (dark points), (b) moderately reactive
			(light points), and (c) nonreactive grains (empty points).}
	\label{fig:filmplots}

\end{figure}
The points are plotted in separate triangles to better illustrate trends in the data. Because of the large number of film grains, with each substrate grain supporting multiple film grains,\footnote{For more information on film growth, see \chapterpageref{polycrystalline.growth}.} plotting all data on the same triangle obscures analysis. When all points are on the same triangle, it becomes difficult to observe individual points. The range of orientations contained in this film does not cover the entire stereographic triangle. \figureref{filmplots} shows the entire range of orientations for the film grains on this sample. Grains that were labeled moderately reactive are scattered throughout the entire orientation spread. No clear orientation preference is observed. When examining the plots for highly reactive and nonreactive grains, the situation becomes more complicated. Highly reactive grains are clustered along axis between the (0001) and (1\={2}13) orientations. Conversely, nonreactive grains are clustered near the (0001) orientation and near the axis between (0001) and (0\={1}12)/(10\={1}2) orientations. This suggests and orientation preference for \ce{Fe2O3} films on polycrystalline \ce{SrTiO3} substrates. This is discussed further in \sectionref{sec:poly.reac.discussion}.


\subsection{Atomic Force Microscopy}
\label{subsec:poly.reac.afm}


After optical microscopy, \abbr{AFM} was then used to confirm the interpretation of the optical image. Five individual areas of the sample surface were scanned, each comprising multiple substrate grains. These areas were selected to contain all reactivity classes and a variety of substrate orientations. Representative \abbr{AFM} micrographs of the examined areas are shown in \figureref{filmafm}.
\begin{figure}
\centering
	\includegraphics[width=0.8\textwidth]{filmafm.pdf}
		\caption[Representative \abbr{AFM} images of film]{%
			Representative \abbr{AFM} images of the film surface after
			reaction, showing highly reactive and nonreactive grains.}
	\label{fig:filmafm}
\end{figure}
Reaction product appears as bright contrast in the \abbr{AFM} images. Varying levels of reactivity are seen in each of the micrographs. In \figureref{filmafm}(a), grains identified as highly reactive are outlined.

The \abbr{AFM} micrographs verify the interpretation of the optical images. Grains that were dark on the optical micrograph were the grains with the largest amount of silver product on the surface. Grains that were bright on the optical image had the least amount of reaction product. These results verify the broader but less detailed results from the optical micrograph, and give increased reliability to the reactivity assignments in \figureref{labeledgrains}. \abbr{AFM} analysis did not show a difference in roughness between grains to account for the order of magnitude reactivity difference observed in these results.


\section{Discussion}
\label{sec:poly.reac.discussion}


Overall, the film is more reactive than the bulk hematite. Grains with orientations that demonstrated low reactivity in the bulk were highly reactive in the film. For similar reaction times, the film showed more reactive grains than the bulk, as well as more reaction product on the surface. Especially considering that the film is only \SI{50}{\nano\meter} thick, and so is only absorbing a small portion of the incident light before the interface with the substrate,\footnote{The penetration depth of the light in this experiment (\textlambda = \si{470}{\nano\meter}), reported in \sectionref{sec:single.crystal.discussion}, is \texttildelow\SI{450}{\nano\meter}, significantly larger than the thickness of the film in this experiment.} the film exhibits high reactivity when compared to bulk \ce{Fe2O3}. Once again, the \ce{SrTiO3} substrate improves the photochemical activity of hematite films when compared to the bulk material.

The observed trends in photochemical reactivity in this chapter represent combined effects from \ce{Fe2O3} orientation and substrate/film interaction, each presented the previous to chapters. In \chapterref{fe2o3orientation}, it was observed that \ce{Fe2O3} reactivity is highly anisotropic, with c-axis oriented crystallites showing low reactivity. Conversely, the c-axis oriented films on \ce{SrTiO3} substrates presented in \chapterref{single.crystal.reactivity} were highly reactive. The films on polycrystalline substrates observed here expose varying film orientations while retaining the orientation relationship between film and substrate observed for the films in \chapterref{single.crystal.reactivity}.\footnote{A complete description of the orientation relationship for \ce{Fe2O3} films on polycrystalline \ce{SrTiO3} substrates can be found in \chapterpageref{polycrystalline.growth}.}.

The optical micrograph presented in \figureref{filmoptical} shows clear signs of reactivity differences between grains. Some grains are dark, completely covered in silver reaction product. Other grains are completely devoid of reaction product. The boundaries between areas of high reactivity and low reactivity correlate with \abbr{EBSD} maps of the substrate grains. Qualitatively, the \ce{Fe2O3} film on the \ce{SrTiO3} substrate is overall significantly more reactive than the polycrystalline sample discussed in \chapterref{fe2o3orientation}. 

The results presented in Figures \ref{fig:substrateplot} and \ref{fig:filmplots} suggest that while the film orientation has an effect on reactivity, the substrate orientation does not. Only the mild clustering of non reactive grains found along the axis between (001) and (101) oriented grains suggests any direct affect of substrate orientation on reactivity. On the other hand, the plots for highly reactive, moderately reactive, and nonreactive films differ from each other. Highly reactive grains are clustered along the center of the standard stereographic triangle, along the axis between (0001) and (1\={2}10). Areas along the edge of the triangle, located at orientations between (0001) and (10\={1}0) do have few active grains. The reverse distribution is seen for the nonreactive grains. Nonreactive grains are clustered near the (0001) orientation, and along the axes between (0001) and (0\={1}10)/(10\={1}0). Grains described as moderately reactive were spread over the entire space of film orientations. 

The results here show similar trends to the anisotropic behavior of bulk hematite crystallites.%
\footnote{%
	The discussion in this chapter is a brief overview of the results for photochemical activity of bulk hematite polycrystals. See \chapterpageref{fe2o3orientation} for a detailed description of the results of this experiment.
} 
It is shown in this document that bulk \ce{Fe2O3} grains located near the (1\={2}10) orientation are the most reactive. Grains far away from this orientation exhibited negligible reactivity. This includes (0001) (c-axis) oriented grains, and also grains at any degree of tilt away from (0001) orientation with the [10\={1}0] direction located out of the sample plane. 

For bulk hematite, the grains that were the most reactive were located relatively near to to the (1\={2}10) orientation. The reactive grains in \figureref{filmplots}(a) are tilted considerably closer to (0001) than any of the reactive grains in that study. However, the results here do follow the general trend observed for that work; hematite grains become reactive as they tilt away from the (0001) orientation, \emph{so long as the [1\={2}10] direction is pointing out of the surface}. In other words, if the axis of rotation away from (0001) is a [10\={1}0] direction, the [1\={2}10] direction will be pointing out of the surface, and the grain is likely to be reactive.\footnote{\figurepageref{fe2o3axes} provides a helpful schematic of the relative orientation of the [10\={1}0] and [1\={2}10] directions and planes in the hematite crystal system.} If the tilt axis is the [1\={2}10] direction, the grain is not likely to be reactive, as the orientation will lie along the nonreactive region between (0001) and (10\={1}0). The results for the film materials show the same trend, however the film becomes reactive and a much lower tilt away from (0001). Compare the distribution of reactive grains in \figureref{filmplots} to the distribution in Figures \ref{fig:semtriangle} or \ref{fig:afmtriangle} in \chapterref{fe2o3orientation}.

For the films on single crystal substrates in \chapterref{single.crystal.reactivity}, (0001)-oriented films on (111)-oriented substrates were highly reactive. Here, many grains close to this substrate orientation were marked as nonreactive or moderately reactive. The contradictory results can be understood through the anisotropic reactivity of hematite. For all \ce{Fe2O3} films on \ce{SrTiO3} substrates, the reactivity of the film was improved compared to the bulk material. When grown on single crystal substrates, the only film orientation observed is (0001)  basal plane oriented \ce{Fe2O3}. This orientation is nonreactive for bulk crystals. When films are grown on polycrystalline substrates, more reactive \ce{Fe2O3} orientations are exposed to the surface. If single crystal substrates existed that could be used to stabilize the reactive faces of \ce{Fe2O3}, it is expected that those films would be even more reactive than the films in the previous chapter. However, the low-index orientations available for \ce{SrTiO3} substrates do not stabilize such films. The results in this chapter suggest that the anisotropic reactivity of \ce{Fe2O3} overwhelms effects from substrate/film interactions.

And interesting phenomenon observed for this experiment is that reactivity in some cases appears to be different between film grains on the same substrate. One set of grains is sometimes more reactive than the other. Reactivity following the lamellar structure of the film grains is observed in multiple instances in \figureref{filmoptical}. These cases were generally those of moderate reactivity, and all film grains on that substrate grain were plotted as mildly reactive in \figureref{filmplots}. It may be that further refinement in the orientation depends of hematite reactivity could be observed through an analysis of these grains. However the detail of the optical image was not enough to accurately determine which grains followed this reactivity pattern. 

Additionally, \figureref{labeledgrains} suggests that grains with similar levels of reactivity are clustered near each other. The highly reactive grains are all located near each other, as are many of the nonreactive grains. This could suggest that the high reactivity on some parts of the sample is not directly related to individual crystallite orientation. However, the boundaries between highly active grains and nonreactive grains do sharply follow the grain boundaries identified using \abbr{EBSD}. The results presented in this document do not entirely prove that another affect could be responsible for the high reactivity and resulting clustering of reactive grains. However, the clear demarcation in reactivity across grain boundaries suggest that there is some affect related to individual crystallite properties.
 



\section{Conclusions}
\label{sec:poly.reac.conclusions}

Hematite films on polycrystalline substrates showed clear signs of reactivity differences between grains. Substrate orientation does not appear to drive this reactivity difference, except in that it also is responsible for resulting film orientation. The film orientation shows a correlation between orientation and high and low reactivity. Highly reactive grains were found in the region between (0001) and (1\={2}13) on the standard stereographic triangle. Nonreactive grains were clustered near (0001) and in the region between (0001) and (10\={1}2)/(01\={1}2). This pattern is similar to the orientation dependent reactivity observed in \chapterref{fe2o3orientation}. Additionally, the film on average was significantly more reactive than bulk hematite. 
