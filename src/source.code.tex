% !TEX root = base.tex 

\chapter{Source Code}

\section{ORs}

The Fortran code below, developed by G.S. Rohrer, was used to calculate orientation relationships between substrate and film orientations using collected EBSD data for films on both single crystal and polycrystalline substrates.

\subsection{main.f}

    \lstset{language=Fortran,
    		stepnumber=2,
    		breaklines=true}

\ttfamily
\begin{lstlisting}

C*****************************************
       program orientation relationships
C*****************************************
       implicit none
	   include 'common.fi'

       write(6,*) '=============================================='
	   write(6,*) ' PROGRAM TO DETERMINE ORIENTATION RELATIONSHIPS'
	   write(6,*) ' BETWEEN A SUBSTRATE AND FILM BASED ON EBSD'
	   write(6,*) ' DATA.  CALCAULATES THE MINIMUM ANGLE BETWEEN'	   
	   write(6,*) ' A REFERENCE DIRECTION IN THE SUBSTRATE AND '	  
	   write(6,*) ' A REFERENCE DIRECTION IN THE FILM'	  
	   write(6,*)
	   write(6,*) '  rohrer@cmu.edu'
	   write(6,*) '  version 11/14/2011'
       write(6,*) '=============================================='
	   
	   version = 'version 11/14/2011'
       pi = 4.0*atan(1.0)

       open(21, file='input.txt', status='old')
	   read(21,*) keyword
c	   read(21,*) phase_in       
	   read(21,*) nref1(1), nref1(2), nref1(3)
	   read(21,*) nref2(1), nref2(2), nref2(3)	 
	   read(21,*) sym1	
	   read(21,*) sym2		   
	   close(21)  
	   point = index(keyword,'.') - 1	

c  This part is to determines the total number of pairs of Euler angles in the data file

		nnline = 0
        open (32, file=keyword(1:point)//'.txt', status='old')
 10     continue
		read(32,*,end=11)
		nnline=nnline+1
		goto 10
 11     continue
        close(32)
 12    	format(A,I7,A)
		write(6,12) 'There are',nnline,' lines in the data file.'

		open(41, file=keyword(1:point)//'.txt', status='old') !pick out file name based on phase
c		if(phase_in.eq.'Anatase') then
	     open(33, file=keyword(1:point)//'_results.txt', status='unknown')
c		else
c        open(33, file=keyword(1:point)//'_results_rutile.txt', status='unknown')
c		endif 
        write(33,20)'file written by ors: ',version ! write informative headers
		write(33,20)'analyzing file: ',keyword
        write(33,22)'Minimum angle between substrate direction: [',nref2(1),', ',nref2(2),', ',nref2(3),']'
        write(33,22)'and film direction: [',nref1(1),', ',nref1(2),', ',nref1(3),'] in degrees'
c		write(33,20)'for :',phase_in
        write(6,20)'written by ors: ',version
		write(6,20)'analyzing file: ',keyword
        write(6,22)'Minimum angle between substrate direction: [',nref2(1),', ',nref2(2),', ',nref2(3),']'
        write(6,22)'and film direction: [',nref1(1),', ',nref1(2),', ',nref1(3),'] in degrees'
c		write(6,20)'for :',phase_in 

 20     format(A,A)	
 22     format(A,f5.3,A,f5.3,A,f5.3,A,A) 
 		
c Now we read the data.  The assumptions about the data are that there are that
c the Euler angles are in radians, and that the first three are from the cubic
c substrate and that the second three are from the tetragonal film. 

        do jj=1,nnline  !Read one line of the data file at a time

		read(41,*) e1(1), e1(2), e1(3), e2(1), e2(2), e2(3), id1,id2
		
c		if (phase.ne.phase_in) then  !Only analyze the phase specified by the input file
c		 goto 8000
c		endif

c  Here we get the symmetry operators for the tetraganal system, and put them in O1, 
c  and the symmetry operators for the cubic system and put them in O2

         call get_symop (sym1, O1, nsymm1)	 !film operators 
	     call get_symop (sym2, O2, nsymm2)	 !substrate operators 

c  This part converts the Euler angles to a g matrix

         call EToG(e1,g2) !substrate orientation
		 call EToG(e2,g1) !film orientation

c   This part finds all the equivalent orientations for the substrate and film
c   Note: I also reversed the crystal order, and the result didn't change

         min_angle=pi
		 min_rot=pi
         do i=1,nsymm1	
		  call symop(O1,i,so1)  !returns the i-th 3x3 tetragonal symmetry operator from O  
		  call MToM (so1,g1,gg1) !hit the film g matrix with the i-th symmetry operator	
		  call trans(gg1,gg1t)  !takes the transpose
		  call MToV (gg1t, nref1, n3) !n3 is [001] direction of the tetragonal crystal, in sample ref frame

          do j=1,nsymm2
		  
		   call symop(O2,j,so2)  !returns the j-th 3x3 cubic symmetry operator from O	   
		   call MToM(so2,g2,gg2)  !hits the substrate g matrix with the j-th symmetry operator	
		   call trans(gg2,gt)  !takes the transpose
		   call MToV (gt, nref2, n4)  !n4 is substrate reference direction, in the sample reference frame
           angle=acos((n3(1)*n4(1))+(n3(2)*n4(2))+(n3(3)*n4(3))) ! angle between the film and substrate reference directions
		   if (angle.lt.min_angle) then
		    min_angle=angle
		   endif	

		  enddo ! end j loop
		 enddo ! end i loop
		  		  
		 write(33,*) min_angle*(180./pi)  !Write to text file
		 write(6,*) min_angle*(180./pi)	  !Write to text terminal		 
		  
 50      format(7f10.2)

c   Now we have to search the list for the minimum misorientation and
c   minimum misorientation between reference directions.

 8000    continue		 
         enddo
		 close(41)
         close(33)
 9000    continue  
         end

  \end{lstlisting}
  
\subsection{sub.f}



  \begin{lstlisting}
      \lstset{language=Fortran,
    		stepnumber=10,
    		breaklines=true}

  c=============================================
	subroutine EToG (euler, g)
c=============================================
	include 'common.fi'
       
	g(1,1)=cos(euler(1))*cos(euler(3))
     &	                 -sin(euler(1))*sin(euler(3))*cos(euler(2))
	g(1,2)=sin(euler(1))*cos(euler(3))
     &	                 +cos(euler(1))*sin(euler(3))*cos(euler(2))
	g(1,3)=sin(euler(3))*sin(euler(2))
	g(2,1)=-cos(euler(1))*sin(euler(3))
     &	                 -sin(euler(1))*cos(euler(3))*cos(euler(2))
	g(2,2)=-sin(euler(1))*sin(euler(3))
     &	                 +cos(euler(1))*cos(euler(3))*cos(euler(2))
	g(2,3)=cos(euler(3))*sin(euler(2))
	g(3,1)=sin(euler(1))*sin(euler(2))
	g(3,2)=-cos(euler(1))*sin(euler(2))
	g(3,3)=cos(euler(2))


	return
	end




c=============================================
	subroutine GToE (delg, euler)
c=============================================
	include 'common.fi'
	
	xz=delg(3,3)
	
	if (xz*xz.lt.1.0) then
	sf=sqrt(1.0-xz*xz)
	else
	sf=0.0
	endif

	if (sf.gt.eps) then
	euler(1)=-delg(3,2)/sf
	euler(1)=acos2(euler(1))
	if (delg(3,1).lt.0.0) euler(1)=2.0*pi-euler(1)
	euler(2)=acos2(xz)
	euler(3)=delg(2,3)/sf
	euler(3)=acos2(euler(3))
	if (delg(1,3).lt.0.0) euler(3)=2.0*pi-euler(3)
	else
	euler(1)=delg(2,2)
	euler(1)=acos2(euler(1))
	if (delg(1,2).lt.0.0) euler(1)=2.0*pi-euler(1)
	euler(2)=0.0
	euler(3)=0.0
	endif


	return
	end




c=============================================
	subroutine MtoM (g1, g2, g3)
c=============================================
	include 'common.fi'

	do ii=1,3
	do jj=1,3
	g3(ii,jj)=0.0
	do kk=1,3
	g3(ii,jj)=g3(ii,jj)+g1(ii,kk)*g2(kk,jj)
	enddo
	enddo
	enddo

	return
	end




c=============================================
	subroutine MToV (g, n1, n2)
c=============================================
    include 'common.fi'

	do ii=1,3
	n2(ii)=0.0
	do jj=1,3
	n2(ii)=n2(ii)+g(ii,jj)*n1(jj)
	enddo
	enddo

	return
	end




c=============================================
	subroutine AnglesToV (theta, psi, vector_out)
c=============================================
	include 'common.fi'
	
	vector_out(1)=0.0
	vector_out(2)=0.0
	vector_out(3)=0.0
	
	vector_out(1)=sin(theta)*cos(psi)
	vector_out(2)=sin(theta)*sin(psi)
	vector_out(3)=cos(theta)

	return
	end




c=============================================
	subroutine VToAngles (v, a)
c=============================================
	include 'common.fi'

	if (v(3).lt.0.0) then
	do ii=1,3
	v(ii)=-v(ii)
	enddo
	endif

	a(1)=acos2(v(3))
c	if (v(1).ne.0.0) then
c	a(2)=atan(v(2)/v(1))
c	else
c	a(2)=pi*0.5
c	endif
	a(2)=atan2(v(2),v(1))
	if (a(2).lt.0.0) a(2)=2.0*pi+a(2)

	return
	end




c=============================================
	subroutine AAToG (x, phi, g)
c=============================================
	include 'common.fi'
	
	g(1,1)=cos(phi)+(1.0-cos(phi))*(x(1)**2)
	g(1,2)=x(1)*x(2)*(1.0-cos(phi))-x(3)*sin(phi)
	g(1,3)=x(1)*x(3)*(1.0-cos(phi))+x(2)*sin(phi)
	g(2,1)=x(1)*x(2)*(1.0-cos(phi))+x(3)*sin(phi)
	g(2,2)=cos(phi)+(1.0-cos(phi))*(x(2)**2)
	g(2,3)=x(3)*x(2)*(1.0-cos(phi))-x(1)*sin(phi)
	g(3,1)=x(1)*x(3)*(1.0-cos(phi))-x(2)*sin(phi)
	g(3,2)=x(2)*x(3)*(1.0-cos(phi))+x(1)*sin(phi)
	g(3,3)=cos(phi)+(1.0-cos(phi))*(x(3)**2)


	return
	end




c=============================================
	subroutine trans (g, gt)
c=============================================
	include 'common.fi'

	do ii=1,3
	do jj=1,3
	gt(ii,jj)=g(jj,ii)
	enddo
	enddo

	return
	end




c=============================================
	subroutine symop (op, i, so)
c=============================================
	include 'common.fi'
	

	do ii=1,3
	do jj=1,3
	so(ii,jj)=op(ii,jj,i)
	enddo
	enddo

	return
	end




c=============================================
      function acos2(value)
c=============================================
	include 'common.fi'
	
      temp=value
	if (temp.gt.1.0) temp=1.0
	if (temp.lt.-1.0) temp=-1.0
	acos2=acos(temp)
	
	return
	end
	  
  \end{lstlisting}

\subsection{symmetry.f}

  \begin{lstlisting}
  
  c=============================================
      subroutine get_symop(ii, O, ns)
c=============================================
       implicit none
	   include 'common.fi'
	
		
		do i = 1,3
           do j = 1,3
              do k = 1,24
                 O(i,j,k) = 0.0
              enddo
           enddo
        enddo


	if (ii.eq.1) then    ! tetragonal (4/mmm)
	ns=8
!     1      
	O(1,1,1)=1.0
	O(1,2,1)=0.0
	O(1,3,1)=0.0
	O(2,1,1)=0.0
	O(2,2,1)=1.0
	O(2,3,1)=0.0
	O(3,1,1)=0.0
	O(3,2,1)=0.0
	O(3,3,1)=1.0
!     2      
	O(1,1,2)=-1.0
	O(1,2,2)=0.0
	O(1,3,2)=0.0
	O(2,1,2)=0.0
	O(2,2,2)=1.0
	O(2,3,2)=0.0
	O(3,1,2)=0.0
	O(3,2,2)=0.0
	O(3,3,2)=-1.0
!     3      
	O(1,1,3)=1.0
	O(1,2,3)=0.0
	O(1,3,3)=0.0
	O(2,1,3)=0.0
	O(2,2,3)=-1.0
	O(2,3,3)=0.0
	O(3,1,3)=0.0
	O(3,2,3)=0.0
	O(3,3,3)=-1.0 !  corrected 10 April 2007, ADR
!     4      
	O(1,1,4)=-1.0
	O(1,2,4)=0.0
	O(1,3,4)=0.0
	O(2,1,4)=0.0
	O(2,2,4)=-1.0
	O(2,3,4)=0.0
	O(3,1,4)=0.0
	O(3,2,4)=0.0
	O(3,3,4)=1.0
!     5      
	O(1,1,5)=0.0
	O(1,2,5)=1.0
	O(1,3,5)=0.0
	O(2,1,5)=-1.0
	O(2,2,5)=0.0
	O(2,3,5)=0.0
	O(3,1,5)=0.0
	O(3,2,5)=0.0
	O(3,3,5)=1.0
!     6      
	O(1,1,6)=0.0
	O(1,2,6)=-1.0
	O(1,3,6)=0.0
	O(2,1,6)=1.0
	O(2,2,6)=0.0
	O(2,3,6)=0.0
	O(3,1,6)=0.0
	O(3,2,6)=0.0
	O(3,3,6)=1.0
!     7      
	O(1,1,7)=0.0
	O(1,2,7)=1.0
	O(1,3,7)=0.0
	O(2,1,7)=1.0
	O(2,2,7)=0.0
	O(2,3,7)=0.0
	O(3,1,7)=0.0
	O(3,2,7)=0.0
	O(3,3,7)=-1.0
!     8      
	O(1,1,8)=0.0
	O(1,2,8)=-1.0
	O(1,3,8)=0.0
	O(2,1,8)=-1.0
	O(2,2,8)=0.0
	O(2,3,8)=0.0
	O(3,1,8)=0.0
	O(3,2,8)=0.0
	O(3,3,8)=-1.0

      elseif (ii.eq.2) then    ! hexagonal (6/mmm)
      ns=12
!     1      
	O(1,1,1)=1.0
	O(1,2,1)=0.0
	O(1,3,1)=0.0
	O(2,1,1)=0.0
	O(2,2,1)=1.0
	O(2,3,1)=0.0
	O(3,1,1)=0.0
	O(3,2,1)=0.0
	O(3,3,1)=1.0
!     2      
	O(1,1,2)=-0.5
	O(1,2,2)=sqrt(3.0)*0.5
	O(1,3,2)=0.0
	O(2,1,2)=-sqrt(3.0)*0.5
	O(2,2,2)=-0.5
	O(2,3,2)=0.0
	O(3,1,2)=0.0
	O(3,2,2)=0.0
	O(3,3,2)=1.0
!     3      
	O(1,1,3)=-0.5
	O(1,2,3)=-sqrt(3.0)*0.5
	O(1,3,3)=0.0
	O(2,1,3)=sqrt(3.0)*0.5
	O(2,2,3)=-0.5
	O(2,3,3)=0.0
	O(3,1,3)=0.0
	O(3,2,3)=0.0
	O(3,3,3)=1.0
!     4      
	O(1,1,4)=0.5
	O(1,2,4)=sqrt(3.0)*0.5
	O(1,3,4)=0.0
	O(2,1,4)=-sqrt(3.0)*0.5
	O(2,2,4)=0.5
	O(2,3,4)=0.0
	O(3,1,4)=0.0
	O(3,2,4)=0.0
	O(3,3,4)=1.0
!     5      
	O(1,1,5)=-1.0
	O(1,2,5)=0.0
	O(1,3,5)=0.0
	O(2,1,5)=0.0
	O(2,2,5)=-1.0
	O(2,3,5)=0.0
	O(3,1,5)=0.0
	O(3,2,5)=0.0
	O(3,3,5)=1.0
!     6      
	O(1,1,6)=0.5
	O(1,2,6)=-sqrt(3.0)*0.5
	O(1,3,6)=0.0
	O(2,1,6)=sqrt(3.0)*0.5
	O(2,2,6)=0.5
	O(2,3,6)=0.0
	O(3,1,6)=0.0
	O(3,2,6)=0.0
	O(3,3,6)=1.0
!     7      
	O(1,1,7)=-0.5
	O(1,2,7)=-sqrt(3.0)*0.5
	O(1,3,7)=0.0
	O(2,1,7)=-sqrt(3.0)*0.5
	O(2,2,7)=0.5
	O(2,3,7)=0.0
	O(3,1,7)=0.0
	O(3,2,7)=0.0
	O(3,3,7)=-1.0
!     8      
	O(1,1,8)=1.0
	O(1,2,8)=0.0
	O(1,3,8)=0.0
	O(2,1,8)=0.0
	O(2,2,8)=-1.0
	O(2,3,8)=0.0
	O(3,1,8)=0.0
	O(3,2,8)=0.0
	O(3,3,8)=-1.0
!     9      
	O(1,1,9)=-0.5
	O(1,2,9)=sqrt(3.0)*0.5
	O(1,3,9)=0.0
	O(2,1,9)=sqrt(3.0)*0.5
	O(2,2,9)=0.5
	O(2,3,9)=0.0
	O(3,1,9)=0.0
	O(3,2,9)=0.0
	O(3,3,9)=-1.0
!     10      
	O(1,1,10)=0.5
	O(1,2,10)=sqrt(3.0)*0.5
	O(1,3,10)=0.0
	O(2,1,10)=sqrt(3.0)*0.5
	O(2,2,10)=-0.5
	O(2,3,10)=0.0
	O(3,1,10)=0.0
	O(3,2,10)=0.0
	O(3,3,10)=-1.0
!     11      
	O(1,1,11)=-1.0
	O(1,2,11)=0.0
	O(1,3,11)=0.0
	O(2,1,11)=0.0
	O(2,2,11)=1.0
	O(2,3,11)=0.0
	O(3,1,11)=0.0
	O(3,2,11)=0.0
	O(3,3,11)=-1.0
!     12      
	O(1,1,12)=0.5
	O(1,2,12)=-sqrt(3.0)*0.5
	O(1,3,12)=0.0
	O(2,1,12)=-sqrt(3.0)*0.5
	O(2,2,12)=-0.5
	O(2,3,12)=0.0
	O(3,1,12)=0.0
	O(3,2,12)=0.0
	O(3,3,12)=-1.0

	elseif (ii.eq.3) then    ! cubic (m-3m)
	ns=24

!     1       h  k  l
	O(1,1,1)=1.0
	O(1,2,1)=0.0
	O(1,3,1)=0.0
	O(2,1,1)=0.0
	O(2,2,1)=1.0
	O(2,3,1)=0.0
	O(3,1,1)=0.0
	O(3,2,1)=0.0
	O(3,3,1)=1.0
c  2		 l  h  k
	O(1,1,2)=0.0
	O(1,2,2)=0.0
	O(1,3,2)=1.0
	O(2,1,2)=1.0
	O(2,2,2)=0.0
	O(2,3,2)=0.0
	O(3,1,2)=0.0
	O(3,2,2)=1.0
	O(3,3,2)=0.0
c  3          k  l  h
	O(1,1,3)=0.0
	O(1,2,3)=1.0
	O(1,3,3)=0.0
	O(2,1,3)=0.0
	O(2,2,3)=0.0
	O(2,3,3)=1.0
	O(3,1,3)=1.0
	O(3,2,3)=0.0
	O(3,3,3)=0.0
c  4	  -k l  -h
	O(1,1,4)=0.0
	O(1,2,4)=-1.0
	O(1,3,4)=0.0
	O(2,1,4)=0.0
	O(2,2,4)=0.0
	O(2,3,4)=1.0
	O(3,1,4)=-1.0
	O(3,2,4)=0.0
	O(3,3,4)=0.0
c  5       -k -l h 
	O(1,1,5)=0.0
	O(1,2,5)=-1.0
	O(1,3,5)=0.0
	O(2,1,5)=0.0
	O(2,2,5)=0.0
	O(2,3,5)=-1.0
	O(3,1,5)=1.0
	O(3,2,5)=0.0
	O(3,3,5)=0.0
c  6	  k  -l -h
	O(1,1,6)=0.0
	O(1,2,6)=1.0
	O(1,3,6)=0.0
	O(2,1,6)=0.0
	O(2,2,6)=0.0
	O(2,3,6)=-1.0
	O(3,1,6)=-1.0
	O(3,2,6)=0.0
	O(3,3,6)=0.0
c  7     -l h  -k 
	O(1,1,7)=0.0
	O(1,2,7)=0.0
	O(1,3,7)=-1.0
	O(2,1,7)=1.0
	O(2,2,7)=0.0
	O(2,3,7)=0.0
	O(3,1,7)=0.0
	O(3,2,7)=-1.0
	O(3,3,7)=0.0
c  8      -l -h k
	O(1,1,8)=0.0
	O(1,2,8)=0.0
	O(1,3,8)=-1.0
	O(2,1,8)=-1.0
	O(2,2,8)=0.0
	O(2,3,8)=0.0
	O(3,1,8)=0.0
	O(3,2,8)=1.0
	O(3,3,8)=0.0
c    9       l  -h -k
	O(1,1,9)=0.0
	O(1,2,9)=0.0
	O(1,3,9)=1.0
	O(2,1,9)=-1.0
	O(2,2,9)=0.0
	O(2,3,9)=0.0
	O(3,1,9)=0.0
	O(3,2,9)=-1.0
	O(3,3,9)=0.0
c 10		 -h k  -l 
	O(1,1,10)=-1.0
	O(1,2,10)=0.0
	O(1,3,10)=0.0
	O(2,1,10)=0.0
	O(2,2,10)=1.0
	O(2,3,10)=0.0
	O(3,1,10)=0.0
	O(3,2,10)=0.0
	O(3,3,10)=-1.0
c 11         -h -k l
	O(1,1,11)=-1.0
	O(1,2,11)=0.0
	O(1,3,11)=0.0
	O(2,1,11)=0.0
	O(2,2,11)=-1.0
	O(2,3,11)=0.0
	O(3,1,11)=0.0
	O(3,2,11)=0.0
	O(3,3,11)=1.0
c 12	      h  -k -l 
	O(1,1,12)=1.0
	O(1,2,12)=0.0
	O(1,3,12)=0.0
	O(2,1,12)=0.0
	O(2,2,12)=-1.0
	O(2,3,12)=0.0
	O(3,1,12)=0.0
	O(3,2,12)=0.0
	O(3,3,12)=-1.0
c 13           -l -k -h 
	O(1,1,13)=0.0
	O(1,2,13)=0.0
	O(1,3,13)=-1.0
	O(2,1,13)=0.0
	O(2,2,13)=-1.0
	O(2,3,13)=0.0
	O(3,1,13)=-1.0
	O(3,2,13)=0.0
	O(3,3,13)=0.0
c 14	       l  -k h  
	O(1,1,14)=0.0
	O(1,2,14)=0.0
	O(1,3,14)=1.0
	O(2,1,14)=0.0
	O(2,2,14)=-1.0
	O(2,3,14)=0.0
	O(3,1,14)=1.0
	O(3,2,14)=0.0
	O(3,3,14)=0.0
c 15           l  k  -h 
	O(1,1,15)=0.0
	O(1,2,15)=0.0
	O(1,3,15)=1.0
	O(2,1,15)=0.0
	O(2,2,15)=1.0
	O(2,3,15)=0.0
	O(3,1,15)=-1.0
	O(3,2,15)=0.0
	O(3,3,15)=0.0
c 16           -l k  h 
	O(1,1,16)=0.0
	O(1,2,16)=0.0
	O(1,3,16)=-1.0
	O(2,1,16)=0.0
	O(2,2,16)=1.0
	O(2,3,16)=0.0
	O(3,1,16)=1.0
	O(3,2,16)=0.0
	O(3,3,16)=0.0 

c 17            -h -l -k 
	O(1,1,17)=-1.0
	O(1,2,17)=0.0
	O(1,3,17)=0.0
	O(2,1,17)=0.0
	O(2,2,17)=0.0
	O(2,3,17)=-1.0
	O(3,1,17)=0.0
	O(3,2,17)=-1.0
	O(3,3,17)=0.0
c 18		    h  -l k  
	O(1,1,18)=1.0
	O(1,2,18)=0.0
	O(1,3,18)=0.0
	O(2,1,18)=0.0
	O(2,2,18)=0.0
	O(2,3,18)=-1.0
	O(3,1,18)=0.0
	O(3,2,18)=1.0
	O(3,3,18)=0.0
c 19          h  l  -k
	O(1,1,19)=1.0
	O(1,2,19)=0.0
	O(1,3,19)=0.0
	O(2,1,19)=0.0
	O(2,2,19)=0.0
	O(2,3,19)=1.0
	O(3,1,19)=0.0
	O(3,2,19)=-1.0
	O(3,3,19)=0.0
c 20	       -h l  k
	O(1,1,20)=-1.0
	O(1,2,20)=0.0
	O(1,3,20)=0.0
	O(2,1,20)=0.0
	O(2,2,20)=0.0
	O(2,3,20)=1.0
	O(3,1,20)=0.0
	O(3,2,20)=1.0
	O(3,3,20)=0.0
c 21          -k -h -l 
	O(1,1,21)=0.0
	O(1,2,21)=-1.0
	O(1,3,21)=0.0
	O(2,1,21)=-1.0
	O(2,2,21)=0.0
	O(2,3,21)=0.0
	O(3,1,21)=0.0
	O(3,2,21)=0.0
	O(3,3,21)=-1.0
c 22	      k  -h l
	O(1,1,22)=0.0
	O(1,2,22)=1.0
	O(1,3,22)=0.0
	O(2,1,22)=-1.0
	O(2,2,22)=0.0
	O(2,3,22)=0.0
	O(3,1,22)=0.0
	O(3,2,22)=0.0
	O(3,3,22)=1.0      ! changed to +1 
c 23          k  h  -l
	O(1,1,23)=0.0
	O(1,2,23)=1.0
	O(1,3,23)=0.0
	O(2,1,23)=1.0
	O(2,2,23)=0.0
	O(2,3,23)=0.0
	O(3,1,23)=0.0
	O(3,2,23)=0.0
	O(3,3,23)=-1.0
c 24           -k h  l
	O(1,1,24)=0.0
	O(1,2,24)=-1.0
	O(1,3,24)=0.0
	O(2,1,24)=1.0
	O(2,2,24)=0.0
	O(2,3,24)=0.0
	O(3,1,24)=0.0
	O(3,2,24)=0.0
	O(3,3,24)=1.0     ! changed to +1

	elseif (ii.eq.4) then    ! trigonal (3/m)
      ns=3
!     1      
	O(1,1,1)=1.0
	O(1,2,1)=0.0
	O(1,3,1)=0.0
	O(2,1,1)=0.0
	O(2,2,1)=1.0
	O(2,3,1)=0.0
	O(3,1,1)=0.0
	O(3,2,1)=0.0
	O(3,3,1)=1.0
!     2      
	O(1,1,2)=cos(2.0/3.0*pi)
	O(1,2,2)=-sin(2.0/3.0*pi)
	O(1,3,2)=0.0
	O(2,1,2)=sin(2.0/3.0*pi)
	O(2,2,2)=cos(2.0/3.0*pi)
	O(2,3,2)=0.0
	O(3,1,2)=0.0
	O(3,2,2)=0.0
	O(3,3,2)=1.0
!     3      
	O(1,1,3)=cos(4.0/3.0*pi)
	O(1,2,3)=-sin(4.0/3.0*pi)
	O(1,3,3)=0.0
	O(2,1,3)=sin(4.0/3.0*pi)
	O(2,2,3)=cos(4.0/3.0*pi)
	O(2,3,3)=0.0
	O(3,1,3)=0.0
	O(3,2,3)=0.0
	O(3,3,3)=1.0
!     4      
	O(1,1,4)=1.0
	O(1,2,4)=0.0
	O(1,3,4)=0.0
	O(2,1,4)=0.0
	O(2,2,4)=1.0
	O(2,3,4)=0.0
	O(3,1,4)=0.0
	O(3,2,4)=0.0
	O(3,3,4)=-1.0
!     5      
	O(1,1,5)=cos(2.0/3.0*pi)
	O(1,2,5)=-sin(2.0/3.0*pi)
	O(1,3,5)=0.0
	O(2,1,5)=sin(2.0/3.0*pi)
	O(2,2,5)=cos(2.0/3.0*pi)
	O(2,3,5)=0.0
	O(3,1,5)=0.0
	O(3,2,5)=0.0
	O(3,3,5)=-1.0
!     6      
	O(1,1,6)=cos(2.0/3.0*pi)
	O(1,2,6)=-sin(2.0/3.0*pi)
	O(1,3,6)=0.0
	O(2,1,6)=sin(2.0/3.0*pi)
	O(2,2,6)=cos(2.0/3.0*pi)
	O(2,3,6)=0.0
	O(3,1,6)=0.0
	O(3,2,6)=0.0
	O(3,3,6)=-1.0


	elseif (ii.eq.5) then    ! orthorhombic (222)
	ns = 4

!     1       h  k  l
	O(1,1,1)=1.0
	O(1,2,1)=0.0
	O(1,3,1)=0.0
	O(2,1,1)=0.0
	O(2,2,1)=1.0
	O(2,3,1)=0.0
	O(3,1,1)=0.0
	O(3,2,1)=0.0
	O(3,3,1)=1.0
c 2		 -h k  -l   180 on Y
	O(1,1,2)=-1.0
	O(1,2,2)=0.0
	O(1,3,2)=0.0
	O(2,1,2)=0.0
	O(2,2,2)=1.0
	O(2,3,2)=0.0
	O(3,1,2)=0.0
	O(3,2,2)=0.0
	O(3,3,2)=-1.0
c 3         -h -k l    180 on Z
	O(1,1,3)=-1.0
	O(1,2,3)=0.0
	O(1,3,3)=0.0
	O(2,1,3)=0.0
	O(2,2,3)=-1.0
	O(2,3,3)=0.0
	O(3,1,3)=0.0
	O(3,2,3)=0.0
	O(3,3,3)=1.0
c 4	      h  -k -l   180 on X
	O(1,1,4)=1.0
	O(1,2,4)=0.0
	O(1,3,4)=0.0
	O(2,1,4)=0.0
	O(2,2,4)=-1.0
	O(2,3,4)=0.0
	O(3,1,4)=0.0
	O(3,2,4)=0.0
	O(3,3,4)=-1.0

      

	endif


	return
	end  
  \end{lstlisting}

\subsection{common.fi}

  \begin{lstlisting}
  
         real pi
	   real O(3,3,24)	   
	   real O1(3,3,24)	  
       real O2(3,3,24)
	   real AA(200,7)
	   real e1(3), e2(3)   
	   real g1(3,3)
	   real g2(3,3)
	   real nref1(3),nref2(3),nref3(3)
	   real nf1(3),nf2(3),nf3(3), ax(3)
	   real disor, angle, min_angle, rot, min_rot
	   real misor
	   real mis1,dis1,mis2,dis2,mis3,dis3
	   real so1(3,3), so2(3,3)
	   real gg1(3,3), gg2(3,3)   
	   real gg1t(3,3), gg2t(3,3)	   
	   real gt(3,3)
       real dg(3,3)
	   real euler(3), angs(2)
	   real g(3,3)	   
	   real xz, xp, yp
	   real delg(3,3)	   
	   real acos2	   
	   real g3(3,3)
	   real n1(3), n2(3), n3(3),n4(4)
       real vector_out(3), theta, psi, phi, v(3), a(2)
	   real x(3)
	   real op(3,3,24)
	   real so(3,3)
	   real temp, value, denom
	   real gamma(62)
	   real tri(12,3)
	   real dihed(3)
	   real delta
	   real xi1, xf1, yi1, yf1
	   real xi2, xf2, yi2, yf2
	   real xi3, xf3, yi3, yf3
       real value1, value2
	   integer nnline, id1, id2
	   integer seed, sym1, sym2
	   integer nsymm, nsymm1, nsymm2, rotation, radian, msym
	   integer i
	   integer j
	   integer k
	   integer ii
	   integer jj
	   integer kk
	   integer ns
	   integer nit, rs_limit
	   integer i_tjs
	   integer counter, point
	   character*11 tempword
	   character*16 version	   
	   character*7 phase
	   character*7 phase_in	   
	   character*100 keyword
	   common/ const1/ pi

       common/ nsymmetry/ nsymm
        
  \end{lstlisting}