% !TEX root = base.tex 

\chapter{Introduction}
\label{ch:introduction}


\section{Motivation}
\label{sec:intro.motivation}


Recent energy crises, along with increased worries about long-term fossil fuel supplies and the negative environmental effects associated with fossil fuel energy have driven research into alternative methods of energy production. One possible alternative, especially for mobile energy applications like powering automobiles, is hydrogen as a fuel source. For hydrogen to become a viable energy source, a new method of sustainable hydrogen production must be developed. The dominant method of industrial hydrogen production utilizes steam reformation of natural gas.\cite{Report:2004wb} This process releases greenhouse gases into the atmosphere, ofsetting the environmental benefits of hydrogen combustion.
A promising clean method of hydrogen production is the photolysis of water. This process uses the energy of solar photons to split water into hydrogen and oxygen gas. Fujishima and Honda\cite{Fujishima:1972hc} showed in 1972 that, under UV illumination, \ce{TiO2} can catalyze water splitting when used as an electrode in an photoelectrochemical cell. Since that discovery, many researchers have worked to develop systems to split water using \ce{TiO2}.\cite{User:2001tg,Frank:1987hd,Karakitsou:1993fq,Linsebigler:1995gi,Schrauzer:1977ex} Major improvements in the performance of photolysis systems are needed before they can compete with steam reformation. The efficiency of current systems is low, owing to charge carrier recombination within the photolysis catalyst and back reaction of intermediate species on the surface. Additionally, many photolysis catalysts are only able to absorb ultraviolet light, which makes up only a small portion of the solar spectrum. This combination of criteria, the need to engineer ways to improve efficiency of photolysis catalysts while also making use of a wider portion of the solar spectrum inspired the research presented in this document.


\section{Research Narrative}
\label{sec:intro.objectives}


This work was guided by the aim of developing further understanding of the effects of polar surfaces and interfaces on photochemical activity. Polar surfaces give rise to electric fields at material surfaces and at interfaces of heterostructures. These electric fields are believed to increase the photochemical activity of the structures by decreasing the rate of charge carrier recombination and increasing charge carrier drift to the surface. The experiments presented in this document test the effects of buried polar surface terminations on the photochemical activity of supported films. This document also represents a drive toward the utilization of visible light for photochemical reactions. Experiments were carried out testing the following questions:

\begin{items}

	\item Do ferroelectric domains in a substrate affect the photochemical activity of supported films, even if the substrate material does not generate charge carriers?
	\item Does spatial selectivity corresponding to ferroelectric domains occur on p-type \ce{BiFeO3} under visible light?
\end{items}

In the process of studying these questions, pulsed laser deposition of hematite \ce{Fe2O3} films was also examined. By growing films on polycrystalline substrates, a much larger array of orientation relationships between the substrate and film were examined. Initial photochemical experiments on hematite films suggested strong substrate effects on film photochemical activity. These observations led to experiments addressing the following questions related to the effect of polar surface terminations, hematite photochemistry, and hematite thin film growth:

\begin{items}
	\item How do hematite films grow on perovskite substrates?
	\item Is it possible to stabilize and characterize epitaxial films on surfaces located far away from low index orientations?
	\item Do polar substrate surface terminations lead to increased reactivity on the surface of supported films?
\end{items}

\section{Hypotheses}\label{sec:intro.hypotheses}
In the context of the stated questions driving this research, the following hypotheses are proposed:

\begin{items}
	\item Charged interfaces increase the photochemical activity of heterostructured systems.
	\item Iron-based ferroelectrics and their heterostructures will show spatially selective reactivity under visible light.
	\item \ce{Fe2O3} films on  single crystal and polycrystalline perovskite substrates will demonstrate a consistent orientation relationship between substrate and film
\end{items}

\section{Approach}
\label{sec:intro.approach}

Hematite phase \ce{Fe2O3} was selected as a visible light, photochemically active catalyst to study the effect of charged surface terminations and ferroelectric domains on the photochemical properties of visible light active films. \ce{Fe2O3} film growth on \ce{SrTiO3} single crystals and randomly oriented polycrystalline \ce{SrTiO3} substrates was characterized via electron backscatter diffraction. Orientation relationships between the substrate and film were examined. Experiments testing the ferroelectric domain selectivity of reactivity of visible light active, p-type \ce{BiFeO3} was motivated by the previously observed spatial selectivity on \ce{BaTiO3}, which is n-type and only reactive under ultraviolet illumination. The photochemical marker reaction of the reduction of aqueous silver ions to neutral silver was used to measure photochemical activity.

\section{Organization}
\label{sec:intro.organization}

This document is organized in five parts. Part I provides an introduction and background to the document, relevant scientific material, and experimental procedures. Part II contains results for experiments testing the photochemical activity of various hematite structures. The bulk reactivity of hematite, and its anisotropic photochemical activity, is reported in \chapterref{fe2o3orientation}. The photochemical activity of hematite films on single crystal and polycrystalline substrates is reported in Chapters \ref{ch:single.crystal.reactivity} and \ref{ch:polycrystalline.reactivity}, respectively. Part III includes results for film growth experiments. \chapterref{single.crystal.growth} presents results for single crystal substrates and \chapterref{polycrystalline.growth} for polycrystalline substrates. Part IV presents the results from early work on \ce{BiFeO3} that inspired the experiments comprising the bulk of this document. Finally, \chapterref{conclusions} summarizes the information in this document, and provides some context for future paths resulting from this work.